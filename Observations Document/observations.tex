\documentclass[reqno]{amsart}
\usepackage{amsmath,amsthm,amssymb,bm,setspace}
\usepackage{hyperref, nameref}
\usepackage{a4wide}
\usepackage{cleveref}
\usepackage{graphicx,color}
\usepackage{tikz}
\usepackage{youngtab}
\usepackage{arydshln}
\usepackage{etoolbox}
\usepackage{biblatex}

\addbibresource{refs.bib}

\usetikzlibrary{positioning,decorations.pathmorphing, decorations.pathreplacing, decorations.shapes}
\numberwithin{equation}{section}
\renewcommand{\baselinestretch}{1.25}

\newtheorem{thm}{Theorem}
\newtheorem{lem}{Lemma}
\newtheorem{prop}[thm]{Proposition}
\newtheorem{cor}[thm]{Corollary}
\newtheorem{remark}[thm]{Remark}

\theoremstyle{definition}
\newtheorem{exam}{Example}
\newtheorem{defn}{Definition}
\newtheorem{conj}{Conjecture}
\newtheorem{obs}{Observation}
\newtheorem{question}{Question}
\newtheorem{problem}{Problem}
\newtheorem*{note}{Note}

\newcommand{\defining}[2]{\hypertarget{#2}{\textit{#1}}}
\newcommand{\defined}[2]{\hyperref{#2}{#1}}

\newcommand\NN{\mathbb{N}}
\newcommand\QQ{\mathbb{Q}}
\newcommand\CC{\mathbb{C}}
\newcommand\ZZ{\mathbb{Z}}
\newcommand\RR{\mathbb{R}}
\newcommand\sn{\mathfrak{S}_n}
\newcommand\Par{\operatorname{Par}}
\newcommand\lm{\lambda / \mu}
\renewcommand\vec[1]{\mathbf{#1}}

\newcommand\PF{\mathcal{P}}
\newcommand\hPF{\overline{\mathcal{P}}}
\newcommand\hPFz{\overline{\mathcal{P}(\ZZ)}}
\newcommand\K{\mathcal{K}}
\newcommand\R{\mathcal{R}}
\newcommand\closure[1]{\overline{#1}}
\newcommand{\seq}[1]{\underline{#1}}

%%%%%%%%%%%%%%%%%%%%%%%%%%%%%%%%%%%%%%%%%%%%%%%%%%%%%%%%%%%%%%%%

\title{Observations on Ribbon-Positive Sequences}

\author{Robert Angarone}
\address{Department of Mathematics \\ University of Minnesota \\ Minneapolis \\ United States}
\email{angar017@umn.edu}

\begin{document}

This document contains observations and questions about the possible relationship between
\begin{itemize}
\item Hilbert series of Koszul algebras,
\item ribbon-positive homomorphisms, and
\item the Hadamard closure of the set of P\'{o}lya Frequency series.
\end{itemize}

\subsection*{Notation}

\begin{itemize}

\item Let $\Lambda$ denote the ring of symmetric functions,
with the usual notation as in EC2.

\item If $f(t) = \sum_{k \geq 0} a_k t^k \in \RR[[t]]$ is a formal power series,
let $\Phi_{f}: \Lambda \to \RR$ denote the ring homomorphism
defined by the rule $h_i \mapsto a_i$.

\item If $A \subseteq \RR[[t]]$, let $A(\ZZ) \subseteq \ZZ[[t]]$ denote the set of series in $A$ with integer coefficients.

\item Let $\PF \subset \RR[[t]]$ denote the set of P\'{o}lya Frequency series.
By the Jacobi-Trudi identity and the Aiseen-Edrei-Schoenberg-Whitney theorem,
this is precisely the set of series $f(t)$ satisfying 
satisfying $\Phi_{f}(1) = 1$ and $\Phi_{f}(s_{\lambda}) \geq 0$
for all $\lambda \in \Par$.

\item Let $\star$ denote the Hadamard product in $\RR[[t]]$, i.e. the product defined by 
\[\left(\sum_{k \geq 0} a_k t^k\right) \star \left(\sum_{k \geq 0} b_k t^k\right) = \sum_{k \geq 0} a_kb_k t^k.\]

\item If $A \subseteq \RR[[t]]$, let $\closure{A}$ denote the closure of $A$ under finite Hadamard products, that is
\[ \closure{A} = \left\{ g_1(t) \star \cdots \star g_\ell(t) : g_1(t), \ldots, g_\ell(t) \in A \right\}. \]

\end{itemize}

\section*{Hilbert Series of Koszul Algebras}

Let $\K$ denote the set of series $f(t)$ with integer coefficients
so that there exists a Koszul algebra $A = \bigoplus_{k \geq 0} A_i$
satisfying $\mathrm{Hilb}(A,t) = f(t)$.

\begin{obs}(Sam-VandeBogert \cite[Theorem~1.1]{SV25})\label{obs:pf-implies-koszul}
Every P\'{o}lya Frequency series with integer coefficients is the Hilbert series of a Koszul algebra.
That is, we have $\PF(\ZZ) \subset \K$. Note that this containment is strict.
\end{obs}

Let $\R$ denote the set of real formal power series $f(t) = \sum_{k \geq 0} a_k t^k$
satisfying $a_0 = 1$ and $\Phi_{f}(s_R) \geq 0$
for all ribbon skew shapes $R$.
We call all such series \defining{ribbon-positive}{def:ribbon-positive}.

\begin{obs}\label{obs:kosuzl-implies-ribbon}
Every Hilbert series of a Koszul algebra is ribbon-positive. That is, we have $\K \subseteq \R(\ZZ)$.
\end{obs}

\begin{proof}[Proof sketch]
I believe this follows from Lemma~4.23 in \cite{VanRibbon} and Remark 4.24 in \cite{SV25}.
\end{proof}

\begin{question}
Is it true that $\R(\ZZ) \subseteq \K$, and thus that $\K = \R(\ZZ)$?
\end{question}

\begin{proof}[Evidence]
On the algebraic level, Lemma~4.23 in \cite{VanRibbon} is a biconditional statement.
Thus it is plausible that \Cref{obs:kosuzl-implies-ribbon} is a biconditional statement at the numerical level,
and thus that $\K = \R(\ZZ)$.
\end{proof}

\begin{proof}[How would you prove it?]
Given a ribbon-positive series, we must find a Koszul algebra with that Hilbert series.
Given a sequence $(a_0, a_1, \ldots)$ which `numerically' satisfies all the relations to be the Hilbert series of a Koszul algebra,
can we impose a Koszul algebra structure on, say, $\bigoplus_{k \geq 0} \QQ^{a_k}$?
\end{proof} 

\section*{Hadamard Closures}

Note that $\PF \subset \closure{\PF}$ is a strict containment.
That is, the set of P\'{o}lya Frequency series is not closed under Hadamard products.
However, the Segre product of graded algebras preserves the Koszul property.
It follosw that $\closure{\K} = \K$. This leads to the following.

\begin{obs}\label{obs:integer-hadamard-closure-in-koszul}
We have $\closure{\PF(\ZZ)} \subseteq \K.$
\end{obs}

\begin{proof}
We have $\PF(\ZZ) \subseteq \K$ and $\closure{\K} = \K$.
Since $\K$ is a Hadamard-closed set containing $\PF(\ZZ)$,
it must contain the Hadamard-closure $\closure{\PF(\ZZ)}$ of $\PF(\ZZ)$.
\end{proof}

Note that if $f(t) \in \closure{\PF(\ZZ)}$ has a prime coefficient in a nontrivial way,
then in fact $f(t) \in \PF(\ZZ)$ (need to add details).
Without having checked for examples, it does seem plausible that there are elements of $\K$ with prime coefficients.
Thus it does \textit{not} seem plausible that $\closure{\PF(\ZZ)} = \K.$

Nevertheless, it could still be true that $\closure{\PF}(\ZZ) = \K$.
In other words, perhaps every Hilbert series of a Koszul algebra can be realized as a finite Hadamard product of P\'{o}lya Frequency series.
The P\'{o}lya Frequency series factors are not required to have integer coefficients individually to lie in $\closure{\PF}(\ZZ)$.
We only require that their Hadamard product has integer coefficients.

Motivated by this possibility, we discuss the possible relationship between $\closure{\PF}$ and $\R$.

\begin{obs}\label{obs:real-hadamard-closure-in-ribbon}
We (probably) have $\closure{\PF} \subseteq \R$.
\end{obs}

\begin{proof}[Proof sketch]
The proof techniques in \cite{AKOS} should readily generalize to show that 
the property $\Phi_{f \star g}(s_R) \geq 0$ holds when $f,g \in \PF$.
Thus the ribbon-positive property is preserved under Hadamard prodcuts of P\'{o}lya Frequency series,
and it follows that $\closure{\PF} \subseteq \R.$
\end{proof}

\begin{question}
Is the set $\R$ Hadamard-closed? That is, do we have $\closure{\R} = \R$?
\end{question}

\begin{proof}[Evidence]
Given my suspicion that $\R(\ZZ) = \K$,
and the fact that $\K$ is Hadamard-closed,
I am inclined to believe this.

Additionally, I think we may be able to generalize the techniques in \cite{AKOS} to show $\Phi_{f \star g}(s_R) \geq 0$ for any $f,g \in \R$,
not just $f,g \in \PF$.
However, it would require more work.
\end{proof}

% \begin{obs}\label{obs:naive-containment}
% We have the following containment relations:
% \begin{itemize}
% \item $\PF(\ZZ) \subset \K$
% \item $\hPFz \subset \K$
% \item $\K \subseteq \R(\ZZ)$
% \end{itemize}
% \end{obs}

% \begin{proof}[Sketch of Proof]
% The first strict containment follows from the fact
% that the set of P\'{o}lya Frequency series is not closed
% under taking Hadamard products.

% The second containment follows from the facts that
% (i) every P\'{o}lya Frequency series
% is the Hilbert series for some Koszul algebra and
% (ii) the set $\K$ is Hadamard-closed.
% In other words, $\K$ is a Hadamard-closed set containing $\PF(\ZZ)$,
% so it must contain $\hPF(\ZZ)$.

% The third containment follows from the work of VandeBogert.
% \end{proof}

\begin{obs}\label{obs:tl-positive-matrices}
A series $f(t) = \sum_{k \geq 0} a_k t^k \in \R$ if and only if every Temperley--Lieb immanant of the matrix
% \[
% \begin{bmatrix}
% a_1 & a_2 & a_3 & a_4 & \cdots & a_n \\
% 1 & a_1 & a_2 & a_3 & \cdots & a_{n-1} \\
% 0 & 1 & a_1 & a_2 & \cdots & a_{n-2} \\
% 0 & 0 & 1 & a_1 & a_2 \\
% 0 & 0 & 0 & 1 & a_1 & \\
% 0 & 0 & 0 & 0 & 1 \\
% \end{bmatrix}
% \]
\[\begin{bmatrix}
a_1 & a_2 &     &        & a_n    \\
1   & a_1 & a_2 &        &        \\
0   &   1 & a_1 & a_2    &        \\
0   &   0 &   1 & a_1    & a_2    \\
0   &  0  &  0  &  1     & a_1    \\
0   &  0  &  0  &   0    &  1     \\
\end{bmatrix}\]
is nonnegative for all $n \geq 1$.
\end{obs}

\begin{proof}[Sketch of proof]
Use \cite[Corollary 3.16]{AKOS} to show that the Temperley--Lieb immanants
of this matrix are precisely the images of ribbons under $\Phi_{f}$.
\end{proof}

\begin{question}
Is it true that $\R \subseteq \closure{\PF}$, and thus that $\R = \closure{\PF}$?
\end{question}

\begin{proof}[How would you prove it?]
Given a ribbon-positive sequence, find a way to express it as a Hadamard product of P\'{o}lya Frequency series.

Here is one way that could potentially work:
\begin{itemize}
\item Use a result of Rhoades and Skandera
to find a planar network $N$, possibly with some negative edge weights,
which realizes the matrix from \Cref{obs:tl-positive-matrices}.
\item Somehow use the hypothesis that all Temperley--Lieb immanants are positive
to create a bijection between path families on $N$
and path families on a tuple $N_1, \ldots, N_k$ of networks
with \textit{positive} edge weights.
\item Observe that taking tuples of path families
corresponds to the Hadamard product of path matrices.
\item Observe that the positive edge weight requirement implies that the path matrices for $N_1, \ldots, N_k$ come from P\'{o}lya Frequency series.
\end{itemize}

% Maybe we could also prove it by expressing the matrix from \Cref{obs:tl-positive-matrices}
% directly as a Hadamard product of totally nonnegative matrices of the same form, then taking a limit at $n \to \infty$.

% For example, when $n=2$, we have the matrix
% \[\begin{bmatrix}
% a & b \\
% 1 & a
% \end{bmatrix}.\]
% In this situation, the requirement that all TL immanants be positive gives $\Phi(1,1)=a^2-b \geq 0$ and $\Phi(2)=b\geq 0$.
% Thus this matrix is actually already totally positive.

% When $n=3$, we have the matrix
% \[\begin{bmatrix}
% a & b & c \\
% 1 & a & b \\
% 0 & 1 & a
% \end{bmatrix}.\]
% The requirement that all TL immanants be positive gives
% \begin{align*}
% \Phi(1,1,1)&=a^3+c-2ab \geq 0 \\ \Phi(2,1)&=ba-c \geq 0 \\
% \Phi(3)&=c \geq 0 \\ \Phi((2,2)/(1))&=ab-c \geq 0.
% \end{align*}
% The $3 \times 3$ matrix would be totally positive if we also had $b^2 - ac \geq 0$.
% It turns out this is forced to be true, for example by observing that
% \[(a^2-b)^2 = a^4 - 2a^2b+b^2 \geq 0,\]
% which implies
% \begin{align*}
% b^2 &\geq 2a^2b-a^4 & (\text{from the above}) \\
% &\geq 2ac - a^4 & (\text{from }) \Phi(2,1) \\
% &\geq 2ac - b^2 & (\text{from }) \Phi(1,1) \\
% \end{align*}
\end{proof}

\section*{Upshot}

If it is indeed true that $\R(\ZZ) = \K$,
then we have a totally  \textit{combinatorial} way of thinking about $\K$, for example using the concatenation/near-concatenation identity.
If $\R(\ZZ) = \K$ \textit{and} it is also true that $\R = \closure{\PF}$,
then we have $\closure{\PF}(\ZZ) = \K$.
This gives a totally \textit{analytic} way of thinking about $\K$.
Of course, by default, we have a way of thinking about $\K$ algebraically.
Perhaps these three persectives in tandem would allow us
to prove new results about $\K$.

\printbibliography

\end{document}